Technologie typu LP-WAN -- rozległe sieci niskiej mocy (ang. \textsl{Low Power Wide Area Networks}) obecnie znajdują
szereg zastosowań przy budowaniu bezprzewodowych sieci czujnikowych, opierając się o~Internet Rzeczy (IoT, ang.
\textsl{Internet of Things}). Są one bardzo wydajną alternatywą dla sieci komórkowych oraz innych tradycyjnych rozwiązań
sieci bezprzewodowych krótkiego zasięgu, które wykorzystywane są w~środowisku IoT, takich jak ZigBee, Bluetooth. Sieci
LPWA (ang. \textsl{Low Power, Wide Area}) oferują unikalny zestaw funkcji -- łączność na dużych zasięgach dla urządzeń
o~niskim poborze mocy i~wymaganej niskiej szybkości transmisji danych, których inne rozwiązania nie są w~stanie
zapewnić.

W~ostatnim czasie pojawiają się różne standardy, które cieszą się szczególnym zainteresowaniem w~przestrzeni technologii
LP-WAN. Najbardziej popularnymi są LoRa (wykorzystując protokół LoRaWAN)\cite{lora-alliance} oraz Sigfox \cite{sigfox},
jednakże są dostępne także rozwiązania alternatywne.

Niniejsza praca ma na celu zbadanie możliwości realizacji oraz parametrów  bezprzewodowej sieci czujnikowej,
wykorzystując do tego standard LoRa. Rozdział \ref{ch:theory} przedstawia teorię związaną ze standardem. Opisuje on jego
teoretyczne możliwości, odniesienie do definicji zawartych w~modelach sieci oraz możliwości związane z~budowaniem sieci,
a~także typowe typologie. Ponadto przedstawia działanie modulacji, która jest podstawą standardu LoRa oraz protokołu
LoRaWAN -- wyznaczanie ramek transmisyjnych, analityczny model przebiegu sygnału, a~także ważne dla modulacji parametry,
które mają wpływ na możliwości sieci.

Rozdział \ref{ch:development} skupia się na krokach, które należało podjąć w~celu implementacji oprogramowania --
jakie narzędzia zostały wykorzystane. Opisuje także, jak praca z~tymi narzędziami wygląda -- podstawowe informacje na
ich temat, uruchamianie projektów oraz zarządzanie elementami oprogramowania.

Szczegóły implementacji, podział na poszczególne elementy przedstawia rozdział \ref{ch:implementation}. Opisuje
wykorzystaną bazę (stm32duino) oraz potrzebne biblioteki, które umożliwiły napisanie kodu źródłowego. Pokazane zostały
także ograniczenia oraz problemy, które zostały napotkane podczas tworzenia oprogramowania. Ponadto przedstawia
poszczególne elementy wchodzące w~skład oprogramowania, które wykorzystane zostało do badań sieci -- elementy składające
się na nawiązanie połączenia między modułami, zbieranie danych z~podłączonego do modułów sensora temperatury i~ciśnienia
oraz transmisji danych do modułu serwera sieciowego. Opisane zostały także funkcje pozwalające na wyznaczanie średniej
kroczącej oraz implementacja strony internetowej wykorzystywanej do przeglądania danych zbieranych przez sieć. Rozdział
\ref{ch:testing} przedstawia wykonane testy, które pozwoliły na określenia czy zaimplementowane oprogramowanie działa
poprawnie.

Aby określić czy możliwe jest zbudowanie prywatnej sieci czujnikowej bazującej na standardzie LoRa, wykonane zostały jej
badania. Przedstawia to rozdział \ref{ch:research}. Sprawdzone zostały podstawowe parametry -- możliwości komunikacji
oraz stopień błędów podczas przesyłania danych. Opisane zostały także przeprowadzona analiza widmowa transmisji między
modułami oraz podjęte próby obserwacji poszczególnych elementów składowych ramki wiadomości w~modulacji LoRa. Ponadto
rozdział przedstawia wykonane badania w~celu określenia skutecznego zasięgu komunikacji, przepustowości sieci, aby móc
porównać je z~wartościami, które podawane są jako możliwości standardu. Dodatkowo opisane zostały wykonane badania
zużycia energii przez moduły sieci, które miały na celu określenie maksymalnego czasu pracy zaimplementowanego
rozwiązania w~przypadku wykorzystania zasilania bateryjnego.