Celem pracy było zaprojektowanie, zaimplementowanie oraz weryfikacja hipotezy -- zbadanie możliwości zbudowania
prywatnej sieci czujnikowej bazującej na standardzie LoRa. W~ramach tego zapoznano się z~podstawami standardu:
działaniem modulacji, elementami składowymi przesyłanej ramki oraz parametrami wpływającymi na transmisję. Opisany
został proces przygotowania środowiska do tworzenia oprogramowania, kroki wymagane do zaczęcia pracy, wykorzystane
biblioteki oraz ograniczenia związane z~wybraną platformą. Ponadto przedstawione zostały elementy składające się na
całość implementowanego oprogramowania -- części dedykowane dla modułów pracujący w~budowanej sieci oraz modułu
wpierającego działanie systemu (serwera sieciowego do prezentacji danych). W~pracy przedstawione zostały także
przeprowadzone badania, które pozwoliły na określenie możliwości budowania sieci czujnikowych opartych o~standard LoRa.

W~ramach części projektowej (implementacji sieci) napisane zostało oprogramowanie, które pozwoliło na komunikację na
płytki rozwojowe STM32 Nucleo64 L152 wyposażone w~moduły LoRa. Oprogramowanie zawiera elementy, które pozwoliły na
weryfikowanie przesyłanych danych, zbieranie pomiarów oraz wyznaczanie z~nich średniej kroczącej, w~celu wyeliminowania
chwilowych znacznych zmian w~otoczeniu pracy sieci. Ponadto zaimplementowane zostało oprogramowanie na moduł serwera
sieciowego, który mógł odbierać dane z~modułu MASTER sieci LoRa. W~tym celu wykorzystana została zakodowana transmisja
przez magistralę I2C, która pozwalała na wykrywanie błędów w~transmisji danych. Zaimplementowana strona internetowa jest
bardzo prostym narzędziem pozwalającym na przeglądanie danych zbieranych przez sieć. Element ten mógłby zostać znacznie
ulepszony przez wykorzystanie innego typu sprzętu. Jednakże wybrana metoda pozwoliła na pokazanie, że nawet bardzo małe
urządzenia, które mogą działać na zasilaniu bateryjnym, mogą zostać z~sukcesem wykorzystane do takiego zadania.

W~ramach badań zweryfikowana została postawiona hipoteza -- czy możliwe jest zbudowanie prywatnej sieci czujnikowej
bazującej na standardzie LoRa. Przeprowadzone badania pozwoliły na zapoznanie się z~wyglądem przesyłanej ramki,
weryfikację działania sieci (w celu sprawdzenia stopnia błędów w~przesyłaniu wiadomości z~danymi), sprawdzenie
skutecznego zasięgu komunikacji oraz przepustowości, a~także poboru prądu oraz możliwości wykorzystania zasilania
bateryjnego do podtrzymania modułów sieci. Otrzymane wyniki badań pozwoliły na określenie, że standard LoRa może zostać
wykorzystany w~przypadku sieci prywatnych, projektowanych na mniejszą skalę.

W~przyszłości możliwe byłoby rozwinięcie oraz usprawnienie działania zaimplementowanego rozwiązania. Jednym z~elementów
byłoby zmiana modułu, sprzętu odpowiedzialnego za funkcję serwera sieciowego do zbierania danych. Możliwe byłoby
zastosowanie rozwiązania wyposażonego w~bazę danych. Pozwoliłoby to na zapisywanie zbieranych danych w~celu
dokładniejszej obróbki oraz bardziej przejrzystej prezentacji ich w~oparciu także o~dane historyczne, np. poprzez
wyświetlanie użytkownikowi wykresów pokazujących zmiany w~czasie. Zmiana wykorzystanego sprzętu mogłaby też pozwolić na
usprawnienie działania sieci czujnikowej -- zwiększenie skutecznego zasięgu komunikacji lub zmniejszenie stopnia błędu
w~transmisjach na większych dystansach.