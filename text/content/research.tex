\section{\label{sect:network-comm}Badanie komunikacji w sieci} Podstawowym badaniem było było sprawdzenie poprawności
komunikacji. Aby otrzymać zestaw danych testowych, moduły pozostały włączone, bez restartowania ich, przez 1 godzinę.
Przy ustawieniu okresu pomiędzy wysyłaniem żądań przez moduł MASTER na 1 minutę dało to 61 pełnych wymian danych w sieci
-- 1 żądanie przy pierwszym uruchomieniu modułu MASTER oraz 60 podczas pracy modułu na zaimplementowanym zegarze (w
sumie 183 żądania do każdego modułu SLAVE). Średnia odległość pomiędzy modułami podczas badania wynosiła 0.45m (45cm).

\begin{table}[!htbp]
    \centering
    \caption{\label{tab:1h-comm-test}Wyniki przeprowadzonego badania komunikacji w czasie 1 godziny.}
    \begin{tabular}{@{}lccc@{}}
        \toprule
        Zapytania         & \multicolumn{1}{l}{SLAVE 1} & \multicolumn{1}{l}{SLAVE 2} & \multicolumn{1}{l}{SLAVE 3} \\ \midrule
        W sumie {[}-{]}   & 183                         & 183                         & 183                         \\
        Nieudane {[}-{]}  & 4                           & 0                           & 1                           \\
        Nieudane {[}\%{]} & 2.19                        & 0                           & 0.55                        \\ \bottomrule
    \end{tabular}
\end{table}

Na podstawie danych sporządzony został wykres, przedstawiający otrzymane wyniki  -- sumę wysłanych zapytań oraz ilość,
która zakończyła się błędem. Przedstawiony on został na rysunku \ref{img:1h-comm-test}.

\begin{figure}[!htbp]
    \centering
    \includegraphics[width=\textwidth]{research/1h-comm-test}
    \caption{\label{img:1h-comm-test}Wykres słupkowy przedstawiający wyniki 1-godzinnego testu komunikacji sieci}
\end{figure}

