Całość oprogramownia wykorzystuje język programowania \texttt{C++}. Projektowana oraz implementowana sieć składa się
z~dwóch typów modułów, stąd też pojawiła się potrzeba zainicjowania dwóch osobnych projektów -- jednego pod elementy
sieci LoRa oraz drugiego, dedykowanego dla modułu webserwera, z~uwagi na zupełnie inną platformę sprzętową. Firmware
napisany został z~wykorzystaniem kilku różnych podejść:
\begin{itemize}[label=--]
    \item modułowego: każdy plik źródłowy odpowiada za zbiór funkcji wykonujących określone zadania (np. praca
          z~biblioteką do modułów LoRa zaimplementowana jest w~pliku \texttt{lora.cpp}),
    \item zorientowanego obiektowo: większość elementów kodu źródłowego jest reprezentowana w~postaci osobnego
          obiektu. Każdy z~nich posiada swoje funkcje oraz pełni określone zadania (np. obiekt \texttt{bme} ma za
          zadanie umożliwić współpracę z~sensorami dostępnymi na płytce czujników BME280, która podłączona jest do
          każdego modułu SLAVE).
\end{itemize}
Ponadto, z~uwagi na wykorzystanie modułów posiadających duże ilości pamięci RAM dostępnej na oprogramowanie,
wykorzystane zostały dostępne w~nowszych wersjach języka \texttt{C++} -- funkcje szablonowe (ang. \textsl{template
    functions}) lub pętle typu \texttt{for-range}. Są to elementy, które znacznie ułatwiły implementację kodu oraz
pozwoliły na minimalizację powtarzalności pewnych elementów.

\section{Framework oraz biblioteki\label{sect:framework-libraries}} Bazą do oprogramowania na wszystkich modułach jest
framework Arduino oraz jego modyfikacja pod platformę STM32 -- stm32duino, która pozwala na wykorzystanie pełnej
funkcjonalności rdzenia Arduino \cite{stm32duino-docs}. Pomimo tego, że biblioteki HAL (ang. \textsl{Hardware
    Abstraction Layer}) oraz framework stm32 są narzędziami dedykowanymi, w~przypadku tego projektu nie można było ich
zastosować. Oryginalna biblioteka do obsługi modułów rozszerzeń LoRa, została wycofana z~użytku na rzecz nowszej
implementacji, pod nowszą wersję płytek Nucleo z~wbudowanym hardware.

\subsection{Wykorzystane biblioteki\label{sect:used-libs}} Do implementacji oprogramowania na wszystkie moduły
wykorzystane zostało kilka bibliotek, które pozwalały na dodanie pełnego zakresu funkcjonalności do każdego z~projektów.

W przypadku bibliotek zewnętrznych (nie będących częścią rdzenia Arduino) były to:
\begin{itemize}[label=--]
    \item STM32duino I-NUCLEO-LRWAN1: biblioteka do uruchomienia oraz pracy z~modułem rozszerzeń LoRa. Pozwala ona na
          pracę w~dwóch trybach: LoRaRadio -- implementacja wykorzystująca tylko standard dolnej warstwy sprzętowej LoRa
          oraz LoRaWAN -- dodająca możliwość podłączenia modułów do istniejącej sieci LoRa oraz wysyłanie i~odbieranie
          z~niej wiadomości,
    \item Adafruit BME280 Library: biblioteka dedykowana do modułów BME280, pozwalająca na zbieranie danych z~sensorów,
          wykorzystując do tego magistralę SPI albo I2C (w zależności od posiadanego modułu rozszerzeń),
    \item Adafruit BusIO: uniwersalna biblioteka dodająca pewien poziom abstrakcji do komunikacji po magistralach I2C
          oraz SPI,
    \item WiFi101: biblioteka, która daje możliwość wykorzystania modułu WiFi obecnego na płytce Adafruit Feather M0
          (wykorzystanej do uruchomienia serwera w~sieci lokalnej).
\end{itemize}
Ponadto, wykorzystane zostały biblioteki I2C oraz SPI, dostępne w~rdzeniu Arduino. Potrzebne były one do uzyskania
komunikacji pomiędzy mikrokontrolerem Adafruit Feather M0 a~modułuem WiFi, sensorami BM280 podłączonymi do modułów SLAVE
oraz do stworzenia połączenia pomiędzy modułem MASTER, a~płytką z~serwerem sieci lokalnej.

\subsection{Ograniczenia związane z~wykorzystaniem Arduino oraz STM32duino\label{sect:framework-limits}} STM32duino,
pomimo tego, że ułatwił, bądź w ogóle pozwolił na pracowanie z~wykorzystywanymi modułami nie jest platformą idealną,
pozbawioną ograniczeń. Jedynym z~nich, które w dość znacznym stopniu utrudniło implementację oprogramowania dla modułów
sieci, był brak przerwań programowych oraz ograniczone możliwości zastosowania przerwań sprzętowych. Stąd też, pojawił
się wymóg zastosowania pewnych obejść, jednocześnie tracą na wydajności implementowanego rozwiązania. Ponadto,
występowały też problemy związane z działaniem magistrali I2C, tutaj w~przypadku modułów Feather oraz standardowego
Arduino -- nie możliwe było wykorzystanie wyświetlacza OLED pracującego na magistrali I2C oraz zarejestrowania samego
mikrokontrolera jako części, z którą można komunikować się po tej magistrali.

\section{Implementacja opgrogramowania elementów sieci\label{sect:firmware-network}}

\subsection{Oprogramowanie modułu MASTER\label{sect:firmware-master}}

\subsection{Oprogramownie modułów SLAVE\label{sect:firmware-slave}}

\section{Implementacja oprogramowania modułu webserwera\label{sect:firmware-webserver}}

