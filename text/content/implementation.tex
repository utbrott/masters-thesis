Całość oprogramowania wykorzystuje język programowania \texttt{C++}. Projektowana oraz implementowana sieć składa się
z~dwóch typów modułów, stąd też pojawiła się potrzeba zainicjowania dwóch osobnych projektów -- jednego pod elementy
sieci LoRa oraz drugiego, dedykowanego dla modułu serwera sieciowego (ang. \textsl{webserver}), z~uwagi na zupełnie inną
platformę sprzętową. Firmware napisany został z~wykorzystaniem kilku różnych podejść:
\begin{itemize}[label=--]
    \item modułowego: każdy plik źródłowy odpowiada za zbiór funkcji wykonujących określone zadania (np. praca
          z~biblioteką do modułów LoRa zaimplementowana jest w~pliku \texttt{lora.cpp}),
    \item obiektowego: większość elementów kodu źródłowego jest reprezentowana w~postaci osobnego obiektu. Każdy
          z~nich posiada swoje funkcje oraz pełni określone zadania (np. obiekt \enquote{\texttt{bme}} ma za zadanie
          umożliwić współpracę z~sensorami dostępnymi na płytce czujników BME280, która podłączona jest do każdego
          modułu SLAVE).
\end{itemize}
Ponadto, wykorzystane zostały elementy języka \texttt{C++}, które dostępne są w~nowszych wersjach -- funkcje szablonowe
(ang. \textsl{template functions}) lub pętle typu \texttt{for-range}. Są to elementy, które znacznie ułatwiły
implementację kodu oraz pozwoliły na minimalizację powtarzalności pewnych elementów.

Z~uwagi na zastosowanie podejścia modułowego, całość oprogramowania składa się z~wielu mniejszych elementów,
podzielonych na odpowiadające im pliki. Aby mieć pewność, że implementowane funkcje nie będą posiadały żadnych kolizji
w~swoich nazwach, zastosowane zostały przestrzenie nazw (ang. \textsl{namespaces}). Co więcej, ponieważ kod źródłowy
jest dostępny w~domenie publicznej (repozytorium na platformie GitHub z~licencją \textsl{MIT} \cite{snyk-sw-license}),
podjęta została decyzja o~dodaniu opisów działania do wszystkich elementów. Wykorzystany został do tego \textsl{Doxygen}
-- narzędzie do generowania dokumentacji (np. formie strony internetowej lub dokumentu w~\LaTeX) na podstawie
specjalnych znaczników w~komentarzach \cite{doxygen}.

\FloatBarrier
\section{Framework oraz biblioteki\label{sect:framework-libraries}} Bazą do oprogramowania na wszystkich modułach jest
framework Arduino oraz jego modyfikacja pod platformę STM32 -- stm32duino, która pozwala na wykorzystanie pełnej
funkcjonalności rdzenia Arduino \cite{stm32duino-docs}. Pomimo tego, że biblioteki HAL (ang. \textsl{Hardware
    Abstraction Layer}) oraz framework STM32 są narzędziami dedykowanymi, w~przypadku tego projektu nie można było ich
zastosować. Oryginalna biblioteka do obsługi modułów rozszerzeń LoRa została wycofana z~użytku na rzecz nowszej
implementacji, pod nowszą wersję płytek Nucleo z~wbudowanym hardware.

\FloatBarrier
\subsection{Wykorzystane biblioteki\label{sect:used-libs}} Do implementacji oprogramowania na wszystkie moduły
wykorzystanych zostało kilka bibliotek, które pozwalały na dodanie pełnego zakresu funkcjonalności do każdego
z~projektów.

W~przypadku bibliotek zewnętrznych (niebędących częścią rdzenia Arduino) były to:
\begin{itemize}[label=--]
    \item STM32duino I-NUCLEO-LRWAN1: biblioteka do uruchomienia oraz pracy z~modułem rozszerzeń LoRa. Pozwala ona na
          pracę w~dwóch trybach: LoRaRadio -- implementacja wykorzystująca tylko standard dolnej warstwy sprzętowej LoRa
          oraz LoRaWAN -- dodająca możliwość podłączenia modułów do istniejącej sieci LoRa oraz wysyłanie i~odbieranie
          z~niej wiadomości,
    \item Adafruit BME280 Library: biblioteka dedykowana do modułów BME280, pozwalająca na zbieranie danych z~sensorów,
          wykorzystując do tego magistralę SPI albo I2C (w~zależności od posiadanego modułu rozszerzeń),
    \item Adafruit BusIO: uniwersalna biblioteka dodająca pewien poziom abstrakcji do komunikacji po magistralach I2C
          oraz SPI,
    \item WiFi101: biblioteka, która daje możliwość wykorzystania modułu WiFi obecnego na płytce Adafruit Feather M0
          (wykorzystanej do uruchomienia serwera w~sieci lokalnej).
\end{itemize}
Ponadto, wykorzystane zostały biblioteki I2C oraz SPI, dostępne w~rdzeniu Arduino. Potrzebne były one do uzyskania
komunikacji pomiędzy mikrokontrolerem Adafruit Feather M0 a~modułem WiFi, sensorami BM280 podłączonymi do modułów SLAVE
oraz do stworzenia połączenia pomiędzy modułem MASTER a~płytką z~serwerem sieci lokalnej.

\FloatBarrier
\subsection{Ograniczenia związane z~wykorzystaniem Arduino oraz STM32duino\label{sect:framework-limits}} STM32duino,
pomimo tego, że ułatwił, bądź w~ogóle pozwolił na pracowanie z~wykorzystywanymi modułami, nie jest platformą idealną,
pozbawioną ograniczeń. Jedynym z~nich, które w~dość znacznym stopniu utrudniło implementację oprogramowania dla modułów
sieci, był brak przerwań programowych oraz ograniczone możliwości zastosowania przerwań sprzętowych. Stąd też pojawił
się wymóg zastosowania pewnych obejść, jednocześnie tracąc na wydajności implementowanego rozwiązania. Ponadto,
występowały też problemy związane z~działaniem magistrali I2C, tutaj w~przypadku modułów Feather oraz standardowego
Arduino -- niemożliwe było wykorzystanie wyświetlacza OLED pracującego na magistrali I2C oraz zarejestrowania samego
mikrokontrolera jako części, z~którą można komunikować się po tej magistrali.

\FloatBarrier
\section{Implementacja oprogramowania elementów sieci\label{sect:firmware-network}} Zaprojektowana sieć składała się
w~sumie z~pięciu modułów -- 4~z~nich stanowiły elementy sieci LoRa, natomiast ostatni był wykorzystywany jako serwer
w~sieci lokalnej. W~projekcie nie została wykorzystana pełna funkcjonalność LoRaWAN oraz typowa dla niej architektura
(przedstawiona w~sekcji \ref{sect:lorawan}, rys. \ref{img:lorawan-architecture}), ponieważ implementacja takiego
rozwiązania jest bardzo kosztowna i~wymaga znacznie większej ilości elementów. Aby móc skorzystać ze specyfikacji
wymagane jest posiadanie bramy (ang. \textsl{gateway}) oraz serwerów odpowiedzialnych za przyłączanie urządzeń,
zarządzanie siecią oraz serwera aplikacyjnego. Z~uwagi na to zastosowana została dużo prostsza i~mniej wymagająca metoda
budowania sieci, opierająca się na wykorzystaniu modułów w~formie nadajników radiowych, pracujących w~standardzie LoRa.
Schemat ideowy budowanej sieci przedstawiony został na rys. \ref{img:network-schematic}.

\begin{figure}[!htbp]
    \centering
    \includegraphics[width=0.8\textwidth]{lora-psn/images/network-schematic}
    \caption{\label{img:network-schematic}Schemat zbudowanej sieci, z~oznaczonymi elementami komunikacji}
\end{figure}

Oprogramowanie dla modułów pracujących w~sieci LoRa zostało zaimplementowane w~formie uniwersalnej -- jeden projekt
zawiera elementy dla modułu MASTER oraz modułów SLAVE. Plik konfiguracyjny projektu zawiera flagę, która definiuje, na
jaki typ modułu kod zostanie skompilowany. Co więcej, w~przypadku modułów SLAVE dodana została też flaga informująca
o~tym, jakie ID przypisane zostaje danej płytce. Rozwiązanie to odgrywa znaczącą rolę w~tym, jak wiadomości są
przesyłane w~sieci. Fragment pliku konfiguracyjnego, który odpowiedzialny jest za definiowanie tych elementów
przedstawiony został na listingu \ref{lst:lora-ini}.

\lstinputlisting[
    linerange={26-30},
    caption={Fragment pliku konfiguracyjnego (tutaj dla SLAVE1) odpowiedzialny za definicję typu oraz ID modułu},
    label={lst:lora-ini},
    float=htbp,
]{lora-psn/platformio.ini}

Wykorzystanie frameworku Arduino wymagało zastosowania pewnych schematów podczas implementacji. Dlatego też całość kodu
podzielona jest na dwie sekcje \texttt{setup()} oraz \texttt{loop()}, wykonywane odpowiednio raz, podczas startu modułu
oraz w~nieskończonej pętli, dopóki płytka ma zasilanie. Na rys. \ref{img:firmware-flowchart} przedstawiony został
schemat blokowy zaimplementowanego oprogramowania -- części zawartej w~sekcji \texttt{setup()}.

\begin{figure}[!htbp]
    \centering
    \includegraphics[width=0.8\textwidth]{lora-psn/images/firmware-flowchart}
    \caption{\label{img:firmware-flowchart}Schemat blokowy części \texttt{setup()} oprogramowania modułów sieci LoRa,
        z~podziałem na typ płytki}
\end{figure}

Oba typy oprogramowania zaczynają od ustawienia portu szeregowego na 115200 baud (szybkość transmisji), następnie
inicjowane jest rozszerzenie LoRa. Logowana jest informacja o~typie płytki, a~następnie kod oczekuje na informacje
o~starcie modułu rozszerzenia. W~przypadku błędu oraz poprawnego startu na port szeregowy wystawiana jest odpowiednia
informacja.

Następnie, w~zależności od typu płytki, wykonywane jest kilka operacji. W~przypadku modułów SLAVE są to:
\begin{enumerate}
    \item przygotowanie sensora BME280 oraz pobranie z~niego danych,
    \item aktualizacja zawartości bufora (wykorzystywanego do przechowywania odczytanych wartości),
    \item przygotowanie diody LED, która informuje o~trwającej komunikacji w~sieci,
    \item przygotowanie przerwania, wykorzystywanego do obsługi nowych zapytań.
\end{enumerate}

Natomiast dla modułów MASTER wykonywany jest inny zestaw operacji, z~uwagi na to, że taki moduł pełni zupełnie inną
funkcję w~sieci:
\begin{enumerate}
    \item przygotowanie tablicy z \enquote{kodami} zapytań (jedno bajtowe wartości do określenia czego żąda MASTER),
    \item inicjacja magistrali I2C i~podłączenie modułu jako MASTER,
    \item wykonanie podprogramu wysyłającego zapytania oraz odbierającego odpowiedzi od SLAVE-ów, tak aby tuż po
          starcie można było odczytać dane z~sieci.
\end{enumerate}
Ostatnim krokiem w~obu przypadkach jest przejście do nieskończonej pętli i~wykonywanie instrukcji w~niej zawartych,
wykorzystując do tego określony okres zegara.

Ponadto, oprogramowanie posiada zestaw definicji oraz funkcji wykorzystywanych do debugowania, które ułatwiały
implementację oprogramowania -- \texttt{globals.h} oraz \texttt{debug.h}. Najważniejszymi elementami pliku globalnych
definicji są funkcje preprocesora -- zwracających tylko ID modułu lub ID danej na podstawie kodu zapytania oraz
struktury szablonowe (ang. \textsl{template structures}), które zawierają informację o~tym jaki kształt powinny mieć
dane zbierane z~sensorów oraz przekazywane przez sieć. Definicję przedstawiono na listingu \ref{lst:globals}, natomiast
dla funkcji wysyłającej sformatowane wiadomości przez port szeregowy na listingu \ref{lst:debug}. Implementacja oparta
została o~funkcję z~rdzenia Arduino -- \texttt{Serial.println()}.

\lstinputlisting[
    language=C++,
    linerange={6-8,15-39},
    caption={Definicje funkcji dla preprocesora oraz struktury szablonowe (z polami o~jednej wartości oraz z~tablicami)},
    label={lst:globals},
    float=htbp,
]{lora-psn/include/globals.h}

\lstinputlisting[
    language=C++,
    linerange={20-26},
    caption={Funkcji wykorzystywana do wysyłania sformatowanych wiadomości przez port szeregowy},
    label={lst:debug},
    float=htbp,
]{lora-psn/include/debug.h}

\FloatBarrier
\subsection{Oprogramowanie modułu MASTER\label{sect:firmware-master}} Po wykonaniu instrukcji, które
opisane zostały w~poprzedniej sekcji, moduł MASTER przechodzi do pracy w~nieskończonej pętli -- \texttt{loop()}.
Wszystko oparte jest na zegarze o~zdefiniowanym okresie -- wybrana została wartość 1~minuty (60000 milisekund).
Implementacja oparta została o~zegar nieblokujący (ang. \textsl{non-blocking timer}) z~wykorzystaniem funkcji
\texttt{millis()} -- funkcji zwracającej ilość milisekund od momentu startu programu. Okres został zdefiniowany
w~definicjach preprocesora, w~celu uniknięcia tzw. magicznych liczb (ang. \textsl{magic numbers}).

W~momencie, gdy mija wymagany czas, program przechodzi do wykonania podprogramu odpowiadającego za wysyłanie zapytań
oraz zbieranie odpowiedzi z~sieci. Na rys. \ref{img:master-flowchart} przedstawiony został diagram blokowy instrukcji
wykonywanych przez moduł MASTER w~nieskończonej pętli oraz tego, co wykonywane jest w~podprogramie komunikacji.
Natomiast pełna implementacja obu tych elementów przedstawiona została na listingach \ref{lst:master-main-loop} oraz
\ref{lst:master-fetch}.

\begin{figure}[!htbp]
    \centering
    \includegraphics[width=\textwidth]{lora-psn/images/MASTER-loop-flowchart}
    \caption{\label{img:master-flowchart}Schemat blokowy nieskończonej pętli oraz podprogramu zbierania danych
        zaimplementowanych dla modułu MASTER}
\end{figure}

\lstinputlisting[
    language=C++,
    linerange={103-110,112-116},
    caption={Implementacja nieskończonej pętli dla modułu MASTER},
    label={lst:master-main-loop},
    float=htbp,
]{lora-psn/src/main.cpp}

\lstinputlisting[
    language=C++,
    linerange={133-174},
    caption={Funkcja podprogramu odpowiedzialnego za zbieranie danych w~sieci},
    label={lst:master-fetch},
    float=htbp,
]{lora-psn/src/main.cpp}

Pierwszym elementem podprogramu jest wysłanie nowego zapytania do sieci -- zaimplementowana została do tego funkcja
\texttt{sendRequest()} zawarta w~przestrzeni nazw \texttt{lora}. Jej implementacja przedstawiona została na listingu
\ref{lst:lora-sendrequest}. Pobierany jest kod, który ma zostać wysłany do sieci, następnie, wykorzystując funkcję
\texttt{debug::println()} logowana jest przesyłana wartość. Korzystając z~funkcji, która dostępna jest w~bibliotece do
obsługi modułu rozszerzeń LoRa, wysyłana jest wiadomość do sieci.

\lstinputlisting[
    language=C++,
    linerange={37-43},
    caption={Implementacja funkcji \texttt{lora::sendRequest()}},
    label={lst:lora-sendrequest},
    float=htbp,
]{lora-psn/src/lora.cpp}

Następnie moduł MASTER oczekuje na odpowiedź od modułu SLAVE, który powinien wysłać odpowiedź. Jeżeli odpowiedź zostana
otrzymana w~ciągu 500ms, następuje przejście do sprawdzenia, czy pierwsze pole odpowiedzi -- identyfikator -- jest
poprawne. Identyfikator zawiera informację o~ID odpowiadającego SLAVE-a oraz ID danej, której wartość jest przesyłana.
W~przeciwnym razie, na port szeregowy przesyłana jest stosowna informacja, a~licznik zapytań z~błędem odpowiedzi jest
zwiększany. Ostatecznie, jeżeli nie wystąpił żaden z~błędów, wykorzystując funkcję \texttt{lora::readResponse()},
odczytana zostaje wartość przesłana w~odpowiedzi. Implementacja funkcji odczytującej przedstawiona została na listingu
\ref{lst:lora-readresponse}.

\lstinputlisting[
    language=C++,
    linerange={79-97},
    caption={Implementacja funkcji odczytującej wartość odpowiedzi modułu SLAVE},
    label={lst:lora-readresponse},
    float=htbp,
]{lora-psn/src/lora.cpp}

W~funkcji sprawdzane są ID modułu, który odpowiedź wysłał oraz ID danej. Na podstawie tej wartości, aktualizowana jest
odpowiednia indeks w~tablicy, która odpowiada polu struktury do przechowywania danych odbieranych z~sieci. Struktura ta
przekazywana jest jako referencja do miejsca w~pamięci poprzez wskaźnik do jej adresu.

Ostatnimi elementami każdej iteracji pętli jest przesłanie zebranych danych przez port szeregowy oraz transmisja danych
do modułu pełniącego funkcję serwera sieciowego. Funkcja logowania danych przez port szeregowy została dodana, po to,
aby było możliwe debugowanie działania oprogramowania oraz naprawa ewentualnie występujących błędów. Do implementacji
wykorzystana została wykorzystana funkcja szablonowa, która pozwoliła na wykorzystanie tego samego fragmentu kodu do
przesyłania wartości z~tablic o~różnym typie zmiennej (\texttt{float} -- zmiennoprzecinkowa -- dla wartości pochodzących
z~sieci oraz \texttt{int} -- liczby całkowite -- dla wartości związanych ze statystykami zapytań). Na funkcje
wykorzystywane do transmisji danych przez magistralę I2C do modułu serwera sieciowego składa się kod zaimplementowany,
korzystając z~tego samego schematu. Przesyłanie wartości z~pojedynczego pola struktury wykorzystuje także funkcję
szablonową, która wywoływana jest kilkukrotnie wewnątrz \texttt{webserverTransmit} w~celu przesłania wszystkich
wymaganych danych. Kod funkcji szablonowych przedstawiony został na listingach \ref{lst:template-log} oraz
\ref{lst:template-transmit}, natomiast implementacja pełnych funkcji do przesyłania danych na listingach
\ref{lst:main-logging} oraz \ref{lst:main-transmitting}. Dodatkowo zaimplementowana została także funkcja pomocnicza do
wyznacznia wartości procentowej zapytań, które zakończyły się błędem. Opiera się ona o~wykonanie dzielenia wartości
z~licznika zapytań z~błędem (\texttt{failedRequests}) przez wartość licznika całkowitej ilości zapytań wysłanych do
każdego z~modułów SLAVE (\texttt{totalRequests}). Kod tej funkcji przedstawiony został na listingu
\ref{lst:main-failedpercent}.

\lstinputlisting[
    language=C++,
    linerange={176-185},
    caption={Implementacja funkcji szablonowej \texttt{logValues()}},
    label={lst:template-log},
    float=htbp,
]{lora-psn/src/main.cpp}

\lstinputlisting[
    language=C++,
    linerange={213-222},
    caption={Implementacja funkcji szablonowej \texttt{transmitPacket()}},
    label={lst:template-transmit},
    float=htbp,
]{lora-psn/src/main.cpp}

\lstinputlisting[
    language=C++,
    linerange={187-211},
    caption={Funkcja wykorzystywana do logowania wartości przez port szeregowy},
    label={lst:main-logging},
    float=htbp,
]{lora-psn/src/main.cpp}

\lstinputlisting[
    language=C++,
    linerange={224-240},
    caption={Funkcja do transmisji danych do modułu serwera przez magistralę I2C},
    label={lst:main-transmitting},
    float=htbp,
]{lora-psn/src/main.cpp}

\lstinputlisting[
    language=C++,
    linerange={242-249},
    caption={Implementacja funkcji pomocnicznej do wyznaczania wartości procentowej zapytań z~błędem},
    label={lst:main-failedpercent},
    float=htbp,
]{lora-psn/src/main.cpp}

\FloatBarrier
\subsection{Oprogramowanie modułów SLAVE\label{sect:firmware-slave}} W~przypadku modułów SLAVE działanie kodu
w~nieskończonej pętli zostało zaimplementowane inaczej. Poza wykonywaniem zadań bazując na zegarze, w~tym przypadku
z~okresem 5~sekund, zaimplementowane zostały także przerwania. Wykorzystywane są one do obsługi przychodzących nowych
zapytań od modułu MASTER. Tak jak zostało to opisane w~sekcji \ref{sect:framework-limits}, we frameworku Arduino nie ma
możliwości wykorzystania przerwań programowych, dlatego też zostało zastosowane obejście bazujące na przerwaniu
sprzętowym. Na rys. \ref{img:slave-flowchart} przedstawiony został diagram nieskończonej pętli zaimplementowanej dla
modułów SLAVE.

\begin{figure}[!htbp]
    \centering
    \includegraphics[width=0.7\textwidth]{lora-psn/images/SLAVE-loop-flowchart}
    \caption{\label{img:slave-flowchart}Diagram blokowy pętli \texttt{loop()} zaimplementowanej dla modułów SLAVE}
\end{figure}


Zadania wykonywane przez moduły, wykorzystując do tego zegar (zaimplementowany w~sposób identyczny do modułu MASTER --
na bazie funkcji \texttt{millis()}), to głównie zbieranie danych z~sensorów płytki BME280 oraz akutualizacja bufora
z~danymi. Implementacja tego fragmentu kodu została przedstawiona na listingu \ref{lst:slave-main-loop}.

\lstinputlisting[
    language=c++,
    linerange={87-101,112-116},
    caption={Implementacja nieskończonej pętli dla modułów SLAVE},
    label={lst:slave-main-loop},
    float=htbp
]{lora-psn/src/main.cpp}

Odczytywanie danych z~sensora odbywa się wykorzystując funkcję \texttt{sensor::readRaw()}, która przedstawiona jest na
listingu \ref{lst:sensor-readraw}. Jedynym jej zadaniem jest odczytanie wszystkich potrzebnych danych oraz dodanie tych
wartości do tablicy dla średniej kroczącej.

\lstinputlisting[
    language=c++,
    linerange={34-39},
    caption={Implementacja funkcji \texttt{sensor::readRaw()} do odczytywania wartości z~sensora BME280},
    label={lst:sensor-readraw},
    float=htbp
]{lora-psn/src/bme280.cpp}

Obliczanie średniej kroczącej rozwiązane zostało poprzez zainicjowanie zaimplementowanej do tego klasy --
\texttt{RollingAvg}. Jej definicja przedstawiona została na listingu \ref{lst:rollingavg-class}. W~celu uzyskania
optymalnego działania kodu -- średnia krocząca jest zadaniem dość intensywnym obliczeniowo -- zastosowane zostało
przydzielanie pamięci (ang. \textsl{memory allocation}) poprzez \texttt{malloc()} oraz wypełnianie tablicy samymi zerami
przy początkowej deklaracji obiektu, dzięki czemu dodawanie wartości jest znacznie szybsze, ponieważ algorytm nie musi
każdorazowo powiększać rozmiaru tablicy. Podczas dodawania wartości do średniej wykonywane jest kilka kroków:
sprawdzenie, czy tablica na pewno istnieje, odjęcie od sumy wartości z~obecnego indeksu oraz zastępowanie jej nową,
zwiększanie indeksu lub zerowanie go, jeżeli wskaźnik dotarł do maksymalnej wartości. Działanie algorytmu przedstawione
zostało na rys. \ref{img:rolling-average}, natomiast na listingu \ref{lst:rollingavg-add} jego implementacja.

\lstinputlisting[
    language=c++,
    linerange={6-22},
    caption={Definicja klasy \texttt{RollingAvg}},
    label={lst:rollingavg-class},
    float=htbp
]{lora-psn/include/rollingavg.h}

\begin{figure}[!htbp]
    \centering
    \includegraphics[width=\textwidth]{schematics/rolling-average}
    \caption{\label{img:rolling-average}Schemat działania algorytmu dodawania nowej wartości do średniej kroczącej dla
        przykładowej tablicy 10 elementów, indeksu 2~oraz nowej wartości 38}
\end{figure}

\lstinputlisting[
    language=c++,
    linerange={36-51},
    caption={Implementacja algorytmu dodającego nową wartość do średniej kroczącej},
    label={lst:rollingavg-add},
    float=htbp
]{lora-psn/src/rollingavg.cpp}

Bufor na dane został dodany w~celu przyspieszenia działania oprogramowania. Dzięki zastosowaniu takiego rozwiązania,
w~momencie pojawienia się nowego zapytania w~sieci, SLAVE, do którego jest ono skierowane, może prawie natychmiast
wysłać odpowiedź. Aktualizacja bufora z~zebranymi danymi (wartością średniej kroczącej) odbywa się poprzez zmianę ich
wartości pól, wykorzystując do tego wskaźniki do obecnych wartości oraz zaimplementowaną funkcję \texttt{getAverage()}
z~klasy \texttt{RollingAvg}. Podczas pobierania wartości średniej pierwszym krokiem jest sprawdzenie, czy tablica
wartości nie jest pusta, następnie zwracana jest wartość -- dzielenie aktualnej sumy przez ilość elementów w~tablicy.
Implementacja tych funkcji przedstawiona została na listingach \ref{lst:sensor-updatebuffer} oraz
\ref{lst:rollingavg-getaverage}.

\lstinputlisting[
    language=c++,
    linerange={41-47},
    caption={Implementacja funkcji aktualizującej zawartość bufora na dane},
    label={lst:sensor-updatebuffer},
    float=htbp
]{lora-psn/src/bme280.cpp}

\lstinputlisting[
    language=c++,
    linerange={57-63},
    caption={Implementacja funkcji do wyznaczania średniej na podstawie obecnej zawartości tablicy średniej kroczącej},
    label={lst:rollingavg-getaverage},
    float=htbp
]{lora-psn/src/rollingavg.cpp}

\FloatBarrier
Obejściem limitów stawianych przez framework Arduino było dodanie przerwania sprzętowego. Pin wbudowanej
diody mikrokontrolerów (\texttt{LED\_BUILTIN}, pin PA5) połączony został z~dowolnym pinem, który nie pełnił żadnej innej
funkcji (tutaj wybrany został pin PB3, zdefiniowany w~kodzie jako \texttt{SLAVE\_INTERRUPT\_PIN}). W~momencie, gdy moduł
rozszerzeń LoRa wykryje nadawaną w~sieci dowolną wiadomość, włączona zostaje dioda, co powoduje wejście w~przerwanie.

Aktywacja przerwania powoduje wywołanie funkcji \texttt{newRequestHandler()}, w~której pierwszym elementem jest
sprawdzenie, czy zaobserwowane w~sieci zapytanie dotyczy danego modułu. Zrealizowane zostało to za pomocą maski, która
pozwala na wydobycie tylko ID modułu z~całego kodu zapytania. Jeżeli kod jest zgodny, to moduł wysyła odpowiedź,
wykorzystując do tego funkcję \texttt{lora::sendResponse()}, a~następnie wyłączana jest dioda. W~przeciwnym przypadku
wyłączona zostanie tylko dioda, a~moduł wyjdzie z~przerwania i~powróci do wykonywania instrukcji z~nieskończonej pętli.
Implementacja funkcji obsługi przerwania przedstawiona została na listingu \ref{lst:main-newrequesthandler}, natomiast
na listingu \ref{lst:lora-sendreponse} przedstawiona została funkcja zajmująca się wysyłaniem odpowiedzi.

\lstinputlisting[
    language=c++,
    linerange={121-131},
    caption={Implementacja funkcji obsługującej przerwania w~modułach SLAVE},
    label={lst:main-newrequesthandler},
    float=htbp
]{lora-psn/src/main.cpp}

\lstinputlisting[
    language=c++,
    linerange={46-76},
    caption={Implementacja funkcji do wysyłania odpowiedzi przez moduły SLAVE},
    label={lst:lora-sendreponse},
    float=htbp
]{lora-psn/src/lora.cpp}

\FloatBarrier
Funkcja \texttt{lora::sendResponse()} także bazuje na wykorzystaniu maski, tym razem do sprawdzenia ID danej jakiej
wartość powinna znajdować się w~odpowiedzi. Każdorazowo wysyłane zostają 3~bajty: echo kodu zapytania -- w~celu
identyfikacji przez moduł MASTER skąd nadeszła odpowiedź oraz 2~bajty zawierających przesyłaną wartość. Dane zbierane
przez sensor są typu float (wartości zmiennoprzecinkowe, 32-bitowe), a~biblioteka może obsługiwać tylko tablice, gdzie
każde pole należy do typu liczb całkowitych o~stałej szerokości 8~bitów (\texttt{uint8\_t}, reprezentuje wartości tylko
dodatnie o~wartości nie większej niż 255 \cite{cpp-fwinteger}). Aby móc swobodnie przesyłać wartości zastosowany został
rzut (ang. \textsl{type casting}) do oryginalnej wielkości do typu \texttt{uint16\_t} oraz podział jej na dwie części,
wykorzystując do tego maski oraz operacje bitowe. Tak przygotowana tablica wysłana jest jako odpowiedź do modułu MASTER.

\FloatBarrier
\section{Implementacja oprogramowania modułu serwera sieciowego\label{sect:firmware-webserver}} Oprogramowanie modułu
serwera sieciowego dzieli się na dwie części: implementację działania serwera sieciowego (ang. \textsl{webserver}) oraz
odbierania i~dekodowania informacji z~modułu MASTER sieci LoRa. Tak jak w~przypadku oprogramowania dla modułów sieci
LoRa, w~przypadku modułu serwera sieciowego na początku jednorazowo wykonywana jest seria instrukcji zawartych w~sekcji
\texttt{loop()}, a~następnie przechodzi do wykonywania instrukcji w~nieskończonej pętli. Schemat działania tej części
kodu przedstawia diagram blokowy na rys. \ref{img:webserver-firmware}, natomiast na listingu
\ref{lst:webserver-firmware} przedstawiona została implementacja tej części oprogramowania.

\begin{figure}[!htbp]
    \centering
    \includegraphics[width=0.35\textwidth]{feather-ws/images/firmware-flowchart}
    \caption{\label{img:webserver-firmware}Diagram blokowy sekcji \texttt{setup()} oprogramowania modułu
        serwera sieciowego}
\end{figure}

\lstinputlisting[
    language=C++,
    linerange={12-20},
    caption={Implementacja funkcji \texttt{setup()} modułu serwera sieciowego},
    label={lst:webserver-firmware},
    float=htbp
]{feather-ws/src/main.cpp}

Pierwszym elementem jest ustawienie port szeregowy na 115200 baud (szybkość transmisji), a~następnie inicjalizacja
magistrali I2C oraz dołączenie modułu w~roli SLAVE (odbiorca). Ostatnimi krokami wykonywanymi przed przejściem do
nieskończonej pętli jest, wykorzystując zaimplementowane funkcje \texttt{wifi::init()} oraz \texttt{webserver::init()},
inicjalizacja modułu WiFi oraz samego serwera sieci lokalnej. Na listingach \ref{lst:wifi-init} oraz
\ref{lst:webserver-init} przedstawione zostały kody źródłowe tych funkcji.

\lstinputlisting[
    language=C++,
    linerange={7-37},
    caption={Implementacja funkcji \texttt{wifi::init()}},
    label={lst:wifi-init},
    float=htbp
]{feather-ws/src/wificonfig.cpp}

\lstinputlisting[
    language=C++,
    linerange={5-5,8-17},
    caption={Implementacja funkcji \texttt{webserver::init()}},
    label={lst:webserver-init},
    float=htbp
]{feather-ws/src/webserver.cpp}

\FloatBarrier
Inicjalizacja WiFi wykorzystuje zdefiniowane w~osobnym pliku konfiguracyjnym SSID (nazwę) oraz hasło do sieci WiFi,
gdzie moduł ma wykonać próbę podłączenia. Do implementacji funkcji została dodana także funkcja sprawdzenia czy
rozszerzenie WiFi na płytce Feather jest poprawnie podłączone. Gdy moduł serwera próbuje zalogować się do sieci,
wbudowana dioda miga, natomiast, gdy połączenie zostanie uzyskane świeci światłem stałym. Ponieważ do poprawnego
działania wymagane jest połączenie z~siecią WiFi, weryfikacja tego kroku została zaimplementowana w~postaci
nieskończonej pętli while, która blokuje dalsze działanie oprogramowania, jeżeli modułowi nie został przydzielony adres
IP w~sieci lokalnej. Implementacja oraz działanie funkcji inicjalizacji serwera sieciowego jest znacznie mniej
skomplikowane -- tworzony jest nowy obiekt serwera \texttt{server}, następnie kod oczekuje na połączenie z~siecią WiFi,
po czym serwer zostaje uruchomiony i~można się do niego podłączyć korzystając z~logowanego przez port szeregowy adresu
IP. Na rys. \ref{img:webserver-website} przedstawiony został zrzut ekranu ze strony internetowej dostępnej w~sieci
lokalnej, gdzie podłączony został moduł serwera sieciowego.

\begin{figure}[!htbp]
    \centering
    \includegraphics[width=\textwidth]{screenshots/webserver-website}
    \caption{\label{img:webserver-website}Wygląd strony internetowej zaimplementowanej do wizualizacji danych zbieranych
        z~sieci}
\end{figure}


Po zakończonej inicjalizacji magistrali I2C, skonfigurowaniu oraz podłączeniu się do WiFi oraz uruchomieniu serwera
oprogramowanie przechodzi do nieskończonej pętli. Zaimplementowane zostały tutaj funkcje odpowiedzialne za serwowanie
strony internetowej dla użytkownika oraz odbieranie danych przez magistralę I2C od modułu MASTER sieci LoRa oraz
dekodowanie ich. Na rys. \ref{img:webserver-flowchart} przedstawiony został schemat blokowy działania obu tych funkcji.

\begin{figure}[!htbp]
    \centering
    \includegraphics[width=\textwidth]{feather-ws/images/loop-flowchart}
    \caption{\label{img:webserver-flowchart}Schemat blokow działania funkcji w~nieskończonej pętli serwera sieciowego}
\end{figure}

\subsection{\label{sect:webserver-website}Wyświetlanie strony internetowej z~danymi} Implementacja serwera sieciowego
oraz strony, do której użytkownik może uzyskać dostęp w~sieci lokalnej bazuje na bibliotece WiFi101. W~każdej iteracji
pętli sprawdzane jest czy do serwera próbuje podłączyć się nowy klient. Jeżeli otrzymane zostanie nowe żądanie (ang.
\textsl{request}), to inicjalizowany jest timer sprawdzjący czy nie nastąpiło przekrocznenie na wysłanie odpowiedzi lub
klient nie stracił połączenia. Wartość ta została przyjęta jako 2~sekundy -- wartość wystarczająco duża, aby połączenie
nie było zamykane zbyt wcześnie oraz wystarczająco niska, aby inne żądania mogły zostać obsłużone bez nadmiernego
oczekiwania. Kod źródłowy tej części przedstawiony został na listingu \ref{lst:wifi-newclient}.

\lstinputlisting[
    language=C++,
    linerange={24-35,58-61},
    caption={Implementacja obsługi nowego żądania w~oprogramowaniu serwera sieciowego},
    label={lst:wifi-newclient},
    float=htbp
]{feather-ws/src/main.cpp}

W~momencie otrzymania nowego żądania serwer odbiera pojedynczo wszystkie przesyłane do niego znaki. Przy każdym znaku
sprawdzane jest czy nie nastąpiło rozłączenie z~klientem. Odbierane znaki są sprawdzane czy jest to znak nowej linii
(\texttt{\textbackslash n}) oraz czy obecnie następna linia jest pusta -- jest to sygnał, że całe zapytanie zostało
odebrane, a~serwer powinien wysłać odpowiedź. W~przypadku, gdy znakiem nie jest \texttt{\textbackslash n}, ani
\texttt{\textbackslash r} zostaje on dodany do obecnej linii, tak aby zebrać pełny nagłówek żądania, zawierający
wszystkie informacje. Implementacja tego elementu przedstawiona została na listingu \ref{lst:wifi-processrequest}.

\lstinputlisting[
    language=C++,
    linerange={42-57},
    caption={Implementacja odbierania przychodzącego żądania},
    label={lst:wifi-processrequest},
    float=htbp
]{feather-ws/src/main.cpp}

\FloatBarrier
Przesyłanie odpowiedzi przez serwer wykonywane wykorzystując do tego zaimplementowaną funkcję
\texttt{webserver::serve()}. Jej implementacja przedstwiona została na listingu \ref{lst:webserver-serve}. W~celu
przesłania odpowiedzi na żądanie oprócz zawartości samej strony wymagane jest przesłanie także nagłówka strony. Aby
zmniejszyć ilość wymaganej pamięci SRAM wymaganej na oprogramowanie znaczna część kodu strony (nagłówek HTTP, nagłówek
strony \texttt{<head>} oraz część wyświetlanej użytkownikowi strony) przechowywana jest w~pamięci programowej
\texttt{PROGMEM} (pamięci flash) mikrokontrolera. Elementy zostają pobrane ze zdefiniowanych zmiennych oraz przesłane do
żądającego klienta wykorzysując do tego pętle typu \texttt{for-range}. Do przesłania elementów, gdzie wymagane są dane,
które moduł odebrał z~sieci LoRa, wykorzystana została zaimplementowana funkcja szablonowa \texttt{printRow}
(przesyłanie jednego wiersza tabeli z~danymi). Kod źródłowy tej funkcji przedstawiony został na listingu
\ref{lst:webserver-printrow}.

\lstinputlisting[
    language=C++,
    linerange={19-44},
    caption={Implementacja funkcji do przesyłania zawartości strony internetowej \texttt{webserver::serve()}},
    label={lst:webserver-serve},
    float=htbp
]{feather-ws/src/webserver.cpp}

\lstinputlisting[
    language=C++,
    linerange={46-64},
    caption={Implementacja funkcji szablonowej \texttt{printRow()}},
    label={lst:webserver-printrow},
    float=htbp
]{feather-ws/src/webserver.cpp}

\FloatBarrier
Funkcja opiera się na modyfikacji ciągu znaków zawierającego format pojedynczej komórki tabeli oraz pętli typu
\texttt{for-range}. Wykorzystuje do tego funkcję \texttt{sprintf}, przyjmując jako argument ciąg, który ma zostać w~to
miejsce wstawiony -- nagłówek wiersza (ciąg znaków, tablicę \texttt{const char}), który zawiera informację o~typie
danych w~danym wierszu lub bezpośrednią wartość otrzymaną z~sieci LoRa (wartość zmiennoprzecinkowa, \texttt{float}).

\FloatBarrier
\subsection{\label{sect:webserver-i2cdecode}Odbieranie oraz dekodowanie danych z~sieci LoRa} W~celu
uzyskania jak najlepszej wydajności odbierania oraz dekodowania otrzywanych danych zaimplementowane rozwiązanie
w~znacznym stopniu ogranicza wykorzystanie funkcji, która powoduje dynamiczny przydział pamięci (ang. \textsl{dynamic
    memory allocation}). Pierwszym elementem jest odebranie pełnej wiadomości. Aby przetwarzanie danych było łatwiejsze
każdy, pojedynczy znak jest dodawany do ciągu znaków (string) -- jest to jedyny element implementacji wykorzystujący
dynamiczną alokację. Odpowiedzialny za to fragment kodu przedstawiony został na listingu \ref{lst:webserver-i2creceive}.

\lstinputlisting[
    language=c++,
    linerange={64-76},
    caption={Implementacja funkcji odbierającej dane przez magistralę I2C},
    label={lst:webserver-i2creceive},
    float=htbp
]{feather-ws/src/main.cpp}

\FloatBarrier
Następnie wywoływana jest funkcja \texttt{decodeStr()}, której kod przedstawiony został na listingu
\ref{lst:webserver-decodestr}. Pierwszym elementem jest przetworzenie wejściowego ciągu znaków na tablicę pojedyncznych
znaków o~długości większej o~jeden znak. Dodatkowy element potrzeby jest dla algorytmu w~celu oznaczenia zakończenia
tablicy. Dodatkowo definiowana jest tablica czteroelementowa do przechowywania indeksu obecnie uzpełnianej danej
\texttt{dataCounter}. Wykorzystując funkcję z~języka C \texttt{strtok()} tablica dzielona jest na znaku \enquote{\&} --
wykorzystanego jako łączenie pomiędzy parami klucz-wartość. Drugim elementem jest podział każdej pary na znaku
\enquote{:} -- łącznika pomiędzy kluczem (od 0~do 4) oraz wartością. Każdorazowo algorytm sprawdza klucz i~na jego
podstawie otrzymana dana jest przypisywana do odpowiedniego pola struktury na indeksie, którego wartość śledzi licznik
(\texttt{dataCounter}). Wszystkie operacje są następnie powtarzane do momentu, gdy algorym dotrze do końca tablicy.

\lstinputlisting[
    language=C++,
    linerange={78-79,84-128},
    caption={Implementacja funkcji do dekodowania danych przesyłanych przez moduł MASTER},
    label={lst:webserver-decodestr},
    float=htbp
]{feather-ws/src/main.cpp}