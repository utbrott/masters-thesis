\streszczeniepracy{ Standard LoRa, należący do technologii typu LPWA (rozległe sieci niskiej mocy, \textsl{Low Power
        Wide Area}), jest obecnie przedmiotem zainteresowań w~przypadku analizy możliwości budowania sieci czujnikowych.
    Oferowane przez niego możliwości -- duże zasięgi oraz niski pobór mocy przez wykorzystywane do implementacji
    urządzenia to jego główne zalety.

    W~celu sprawdzenia, czy możliwe jest wykorzystanie standardu LoRa do zbudowania prywatnej sieci czujnikowej,
    pozwalającej na zbieranie danych o~otoczeniu (temperatura, ciśnienie, wilgotność) zostało zaprojektowane oraz
    zaimplementowane rozwiązanie, które wykorzystane zostało do weryfikacji takiego założenia.

    Niniejsza praca przedstawia informacje potrzebne do zapoznania się ze standardem oraz podstawami jego działania.
    Opisane zostały działanie modulacji oraz parametry wpływające decydujące o~działaniu sieci i~możliwościach
    transmisji. Przedstawione zostały także kroki, które wymagane były do implementacji oprogramowania działającego na
    modułach. Zaimplementowane rozwiązanie wykorzystuje język C++ oraz płytki rozwojowe STM32 Nucleo64 L152.
    Przedstawione zostały elementy kodu źródłowego odpowiadającego za poszczególne elementy działania oprogramowania --
    komunikacja między modułami, zbieranie danych z~otoczenia oraz wyznaczanie ich średniej kroczącej, a~także
    transmisja ich przez magistralę I2C do modułu serwera sieciowego. Ponadto przedstawiona została metoda dekodowania
    informacji oraz prezentacjach ich użytkownikowi w~formie strony internetowej dostępnej w~sieci lokalnej. Pokazane
    zostały także przeprowadzone testy, które pozwoliły na sprawdzenie poprawności działania zaimplementowanego
    rozwiązania. Pełny kod źródłowy dostępny jest w~przestrzeni publicznej, dzięki temu możliwe jest dalsze rozwijanie
    go.

    Przeprowadzone zostały badania, które miały na celu określenie potencjału zaprojektowanej sieci. Przeanalizowane
    zostały widmo częstotliwości w~trakcie transmisji danych w~sieci, co pozwoliło na dokładniejsze zapoznanie się ze
    standardem. Ponadto zweryfikowane zostały skuteczne zasięgi komunikacji oraz przepustowość sieci, jak i~zużycie
    energii podczas pracy modułów. Zebrane dane pozwoliły na stwierdzenie, że możliwe jest zbudowanie prywatnej sieci
    czujnikowej, bazując na standardzie LoRa. }

\slowakluczowe{
    sieci czujnikowe, LoRa, IoT, Internet Rzeczy, oprogramowanie wbudowane, STM32, mikrokontrolery
}

\thesisabstract{The LoRa standard, which belongs to LPWA (\textsl{Low Power Wide Area}) technology, is currently of
    interest when analysing the feasibility of building sensor networks. The capabilities it offers -- long ranges and
    low power consumption of the devices used for implementation devices are its main advantages.

    In order to test whether it is possible to use the LoRa standard to build a~private sensor network, allowing the
    collection of ambient data (temperature, pressure, humidity), a~solution was designed and implemented, which was
    used to verify such an assumption.

    This work presents the information needed to learn about the standard and the basics of its operation. The operation
    of the modulation and the influencing parameters determining the network operation and transmission capabilities are
    described. The steps required to implement the software running on the modules are also presented. The implemented
    solution uses the C++ language and the STM32 Nucleo64 L152 development boards. The elements of the source code
    responsible for the various elements of the software operation -- communication between the modules, collecting data
    from the environment and determining its moving average, and transmitting it via the I2C bus to the network server
    module -- are presented. In addition, the method of decoding the information and presenting it to the user in
    the~form of a~web page accessible on the~local network is presented. Testing done to verify the correct operation of
    the implemented solution is also shown. The full source code is available in the public domain, so that it can be
    further developed.

    Tests were carried out to determine the potential of the designed network. The frequency spectrum during data
    transmission over the network was analysed, allowing a~more detailed understanding of the standard. In addition, the
    effective communication ranges and network throughput were verified, as well as the energy consumption during module
    operation. The data collected made it possible to conclude that it is possible to build a~private sensor network
    based on the LoRa standard. }

\thesiskeywords{
    sensor networks, LoRa, IoT, Internet of Things, embedded firmware+, STM32, microcontrollers
}